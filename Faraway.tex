% Options for packages loaded elsewhere
\PassOptionsToPackage{unicode}{hyperref}
\PassOptionsToPackage{hyphens}{url}
%
\documentclass[
]{article}
\usepackage{lmodern}
\usepackage{amssymb,amsmath}
\usepackage{ifxetex,ifluatex}
\ifnum 0\ifxetex 1\fi\ifluatex 1\fi=0 % if pdftex
  \usepackage[T1]{fontenc}
  \usepackage[utf8]{inputenc}
  \usepackage{textcomp} % provide euro and other symbols
\else % if luatex or xetex
  \usepackage{unicode-math}
  \defaultfontfeatures{Scale=MatchLowercase}
  \defaultfontfeatures[\rmfamily]{Ligatures=TeX,Scale=1}
  \setmainfont[]{NanumGothic}
\fi
% Use upquote if available, for straight quotes in verbatim environments
\IfFileExists{upquote.sty}{\usepackage{upquote}}{}
\IfFileExists{microtype.sty}{% use microtype if available
  \usepackage[]{microtype}
  \UseMicrotypeSet[protrusion]{basicmath} % disable protrusion for tt fonts
}{}
\makeatletter
\@ifundefined{KOMAClassName}{% if non-KOMA class
  \IfFileExists{parskip.sty}{%
    \usepackage{parskip}
  }{% else
    \setlength{\parindent}{0pt}
    \setlength{\parskip}{6pt plus 2pt minus 1pt}}
}{% if KOMA class
  \KOMAoptions{parskip=half}}
\makeatother
\usepackage{xcolor}
\IfFileExists{xurl.sty}{\usepackage{xurl}}{} % add URL line breaks if available
\IfFileExists{bookmark.sty}{\usepackage{bookmark}}{\usepackage{hyperref}}
\hypersetup{
  pdftitle={Faraway},
  pdfauthor={Harry Woo},
  hidelinks,
  pdfcreator={LaTeX via pandoc}}
\urlstyle{same} % disable monospaced font for URLs
\usepackage[margin=1in]{geometry}
\usepackage{color}
\usepackage{fancyvrb}
\newcommand{\VerbBar}{|}
\newcommand{\VERB}{\Verb[commandchars=\\\{\}]}
\DefineVerbatimEnvironment{Highlighting}{Verbatim}{commandchars=\\\{\}}
% Add ',fontsize=\small' for more characters per line
\usepackage{framed}
\definecolor{shadecolor}{RGB}{248,248,248}
\newenvironment{Shaded}{\begin{snugshade}}{\end{snugshade}}
\newcommand{\AlertTok}[1]{\textcolor[rgb]{0.94,0.16,0.16}{#1}}
\newcommand{\AnnotationTok}[1]{\textcolor[rgb]{0.56,0.35,0.01}{\textbf{\textit{#1}}}}
\newcommand{\AttributeTok}[1]{\textcolor[rgb]{0.77,0.63,0.00}{#1}}
\newcommand{\BaseNTok}[1]{\textcolor[rgb]{0.00,0.00,0.81}{#1}}
\newcommand{\BuiltInTok}[1]{#1}
\newcommand{\CharTok}[1]{\textcolor[rgb]{0.31,0.60,0.02}{#1}}
\newcommand{\CommentTok}[1]{\textcolor[rgb]{0.56,0.35,0.01}{\textit{#1}}}
\newcommand{\CommentVarTok}[1]{\textcolor[rgb]{0.56,0.35,0.01}{\textbf{\textit{#1}}}}
\newcommand{\ConstantTok}[1]{\textcolor[rgb]{0.00,0.00,0.00}{#1}}
\newcommand{\ControlFlowTok}[1]{\textcolor[rgb]{0.13,0.29,0.53}{\textbf{#1}}}
\newcommand{\DataTypeTok}[1]{\textcolor[rgb]{0.13,0.29,0.53}{#1}}
\newcommand{\DecValTok}[1]{\textcolor[rgb]{0.00,0.00,0.81}{#1}}
\newcommand{\DocumentationTok}[1]{\textcolor[rgb]{0.56,0.35,0.01}{\textbf{\textit{#1}}}}
\newcommand{\ErrorTok}[1]{\textcolor[rgb]{0.64,0.00,0.00}{\textbf{#1}}}
\newcommand{\ExtensionTok}[1]{#1}
\newcommand{\FloatTok}[1]{\textcolor[rgb]{0.00,0.00,0.81}{#1}}
\newcommand{\FunctionTok}[1]{\textcolor[rgb]{0.00,0.00,0.00}{#1}}
\newcommand{\ImportTok}[1]{#1}
\newcommand{\InformationTok}[1]{\textcolor[rgb]{0.56,0.35,0.01}{\textbf{\textit{#1}}}}
\newcommand{\KeywordTok}[1]{\textcolor[rgb]{0.13,0.29,0.53}{\textbf{#1}}}
\newcommand{\NormalTok}[1]{#1}
\newcommand{\OperatorTok}[1]{\textcolor[rgb]{0.81,0.36,0.00}{\textbf{#1}}}
\newcommand{\OtherTok}[1]{\textcolor[rgb]{0.56,0.35,0.01}{#1}}
\newcommand{\PreprocessorTok}[1]{\textcolor[rgb]{0.56,0.35,0.01}{\textit{#1}}}
\newcommand{\RegionMarkerTok}[1]{#1}
\newcommand{\SpecialCharTok}[1]{\textcolor[rgb]{0.00,0.00,0.00}{#1}}
\newcommand{\SpecialStringTok}[1]{\textcolor[rgb]{0.31,0.60,0.02}{#1}}
\newcommand{\StringTok}[1]{\textcolor[rgb]{0.31,0.60,0.02}{#1}}
\newcommand{\VariableTok}[1]{\textcolor[rgb]{0.00,0.00,0.00}{#1}}
\newcommand{\VerbatimStringTok}[1]{\textcolor[rgb]{0.31,0.60,0.02}{#1}}
\newcommand{\WarningTok}[1]{\textcolor[rgb]{0.56,0.35,0.01}{\textbf{\textit{#1}}}}
\usepackage{graphicx,grffile}
\makeatletter
\def\maxwidth{\ifdim\Gin@nat@width>\linewidth\linewidth\else\Gin@nat@width\fi}
\def\maxheight{\ifdim\Gin@nat@height>\textheight\textheight\else\Gin@nat@height\fi}
\makeatother
% Scale images if necessary, so that they will not overflow the page
% margins by default, and it is still possible to overwrite the defaults
% using explicit options in \includegraphics[width, height, ...]{}
\setkeys{Gin}{width=\maxwidth,height=\maxheight,keepaspectratio}
% Set default figure placement to htbp
\makeatletter
\def\fps@figure{htbp}
\makeatother
\setlength{\emergencystretch}{3em} % prevent overfull lines
\providecommand{\tightlist}{%
  \setlength{\itemsep}{0pt}\setlength{\parskip}{0pt}}
\setcounter{secnumdepth}{-\maxdimen} % remove section numbering
\usepackage{amsmath}
\usepackage{booktabs}
\usepackage{caption}
\usepackage{longtable}

\title{Faraway}
\author{Harry Woo}
\date{2020-5-12}

\begin{document}
\maketitle

This is an R Markdown document. Markdown is a simple formatting syntax
for authoring HTML, PDF, and MS Word documents. For more details on
using R Markdown see \url{http://rmarkdown.rstudio.com}.

\hypertarget{chapter-2-estimation}{%
\subsection{Chapter 2 Estimation}\label{chapter-2-estimation}}

\hypertarget{problem-1}{%
\subsection{Problem 1}\label{problem-1}}

\emph{The dataset teengamb concerns a study of teenage gambling in
Britain. Fit a regression model with the expenditure on gambling as the
response and the sex, status, income and verbal score as predictors.
Present the output.}

\begin{Shaded}
\begin{Highlighting}[]
\KeywordTok{data}\NormalTok{(teengamb)}
\KeywordTok{head}\NormalTok{(teengamb)}
\end{Highlighting}
\end{Shaded}

\begin{verbatim}
##   sex status income verbal gamble
## 1   1     51   2.00      8    0.0
## 2   1     28   2.50      8    0.0
## 3   1     37   2.00      6    0.0
## 4   1     28   7.00      4    7.3
## 5   1     65   2.00      8   19.6
## 6   1     61   3.47      6    0.1
\end{verbatim}

\begin{Shaded}
\begin{Highlighting}[]
\NormalTok{tg_lm <-}\StringTok{  }\KeywordTok{lm}\NormalTok{(gamble }\OperatorTok{~}\StringTok{ }\NormalTok{sex }\OperatorTok{+}\StringTok{ }\NormalTok{status }\OperatorTok{+}\StringTok{ }\NormalTok{income }\OperatorTok{+}\StringTok{ }\NormalTok{verbal, }\DataTypeTok{data =}\NormalTok{ teengamb)}
\NormalTok{tg_lms <-}\StringTok{ }\KeywordTok{summary}\NormalTok{(tg_lm)}
\KeywordTok{print}\NormalTok{(tg_lms)}
\end{Highlighting}
\end{Shaded}

\begin{verbatim}
## 
## Call:
## lm(formula = gamble ~ sex + status + income + verbal, data = teengamb)
## 
## Residuals:
##     Min      1Q  Median      3Q     Max 
## -51.082 -11.320  -1.451   9.452  94.252 
## 
## Coefficients:
##              Estimate Std. Error t value Pr(>|t|)    
## (Intercept)  22.55565   17.19680   1.312   0.1968    
## sex         -22.11833    8.21111  -2.694   0.0101 *  
## status        0.05223    0.28111   0.186   0.8535    
## income        4.96198    1.02539   4.839 1.79e-05 ***
## verbal       -2.95949    2.17215  -1.362   0.1803    
## ---
## Signif. codes:  0 '***' 0.001 '**' 0.01 '*' 0.05 '.' 0.1 ' ' 1
## 
## Residual standard error: 22.69 on 42 degrees of freedom
## Multiple R-squared:  0.5267, Adjusted R-squared:  0.4816 
## F-statistic: 11.69 on 4 and 42 DF,  p-value: 1.815e-06
\end{verbatim}

\hypertarget{a-what-percentage-of-variation-in-the-response-is-explained-by-these-predictors}{%
\subsubsection{(a) What percentage of variation in the response is
explained by these
predictors?}\label{a-what-percentage-of-variation-in-the-response-is-explained-by-these-predictors}}

전체 제곱합(SST)에서 회귀 제곱합(SSR)이 설명하는 비중, 즉 모형의
설명력은 결정 계수 R\textsuperscript{2} 이다. 위 Summary 에서와 같이 동
모형의 결정계수 \textbf{Multiple R-squared = 0.5267} 이다.

\begin{Shaded}
\begin{Highlighting}[]
\NormalTok{var_ex <-}\StringTok{ }\KeywordTok{data.frame}\NormalTok{(}\DataTypeTok{Var_explained =}\NormalTok{ tg_lms}\OperatorTok{$}\NormalTok{r.squared)}
\NormalTok{var_ex }\OperatorTok\StringTok{ }\KeywordTok{gt}\NormalTok{() }\OperatorTok\StringTok{ }
\StringTok{  }\KeywordTok{fmt_percent}\NormalTok{(}\DataTypeTok{columns =} \KeywordTok{vars}\NormalTok{(Var_explained),}
              \DataTypeTok{decimals =} \DecValTok{2}\NormalTok{)}
\end{Highlighting}
\end{Shaded}

\captionsetup[table]{labelformat=empty,skip=1pt}
\begin{longtable}{r}
\toprule
Var\_explained \\ 
\midrule
$52.67\%$ \\ 
\bottomrule
\end{longtable}

\hypertarget{b-which-observation-has-the-largest-positive-residual-give-the-case-number.}{%
\subsubsection{(b) Which observation has the largest (positive)
residual? Give the case
number.}\label{b-which-observation-has-the-largest-positive-residual-give-the-case-number.}}

회귀모형의 \texttt{residuals} 를 데이터프레임으로 변환하여 잔차값
기준으로 내림차순 정렬을 시행해 largest residual의 case number를 추출한
결과, \textbf{해당 case number는 24} 이다.

\begin{Shaded}
\begin{Highlighting}[]
\NormalTok{res <-}\StringTok{ }\KeywordTok{data.frame}\NormalTok{(}\DataTypeTok{case_no =} \KeywordTok{c}\NormalTok{(}\DecValTok{1}\OperatorTok{:}\DecValTok{47}\NormalTok{), }\DataTypeTok{residual =}\NormalTok{ tg_lm}\OperatorTok{$}\NormalTok{residuals)}
\NormalTok{res }\OperatorTok\StringTok{ }
\StringTok{  }\KeywordTok{arrange}\NormalTok{(}\KeywordTok{desc}\NormalTok{(residual)) }\OperatorTok\StringTok{ }
\StringTok{  }\KeywordTok{slice}\NormalTok{(}\DecValTok{1}\NormalTok{) }\OperatorTok\StringTok{ }
\StringTok{  }\KeywordTok{gt}\NormalTok{()}
\end{Highlighting}
\end{Shaded}

\captionsetup[table]{labelformat=empty,skip=1pt}
\begin{longtable}{cr}
\toprule
case\_no & residual \\ 
\midrule
24 & 94.25222 \\ 
\bottomrule
\end{longtable}

\hypertarget{c-compute-the-mean-and-median-of-the-residuals.}{%
\subsubsection{(c) Compute the mean and median of the
residuals.}\label{c-compute-the-mean-and-median-of-the-residuals.}}

.

\begin{Shaded}
\begin{Highlighting}[]
\NormalTok{res }\OperatorTok\StringTok{ }
\StringTok{  }\KeywordTok{summarise}\NormalTok{(}\DataTypeTok{mean =} \KeywordTok{mean}\NormalTok{(residual), }\DataTypeTok{median =} \KeywordTok{median}\NormalTok{(residual)) }\OperatorTok\StringTok{ }
\StringTok{  }\KeywordTok{gt}\NormalTok{()}
\end{Highlighting}
\end{Shaded}

\captionsetup[table]{labelformat=empty,skip=1pt}
\begin{longtable}{rr}
\toprule
mean & median \\ 
\midrule
-3.065293e-17 & -1.451392 \\ 
\bottomrule
\end{longtable}

\hypertarget{d-compute-the-correlation-of-the-residuals-with-the-fitted-values.}{%
\subsubsection{(d) Compute the correlation of the residuals with the
fitted
values.}\label{d-compute-the-correlation-of-the-residuals-with-the-fitted-values.}}

.

\begin{Shaded}
\begin{Highlighting}[]
\KeywordTok{data.frame}\NormalTok{(}\DataTypeTok{correlation =} \KeywordTok{cor}\NormalTok{(tg_lm}\OperatorTok{$}\NormalTok{residuals, tg_lm}\OperatorTok{$}\NormalTok{fitted.values)) }\OperatorTok\StringTok{ }
\StringTok{  }\KeywordTok{gt}\NormalTok{()}
\end{Highlighting}
\end{Shaded}

\captionsetup[table]{labelformat=empty,skip=1pt}
\begin{longtable}{r}
\toprule
correlation \\ 
\midrule
-1.070659e-16 \\ 
\bottomrule
\end{longtable}

\hypertarget{e-compute-the-correlation-of-the-residuals-with-the-income.}{%
\subsubsection{(e) Compute the correlation of the residuals with the
income.}\label{e-compute-the-correlation-of-the-residuals-with-the-income.}}

.

\begin{Shaded}
\begin{Highlighting}[]
\KeywordTok{data.frame}\NormalTok{(}\DataTypeTok{correlation =} \KeywordTok{cor}\NormalTok{(teengamb}\OperatorTok{$}\NormalTok{income, tg_lm}\OperatorTok{$}\NormalTok{fitted.values)) }\OperatorTok\StringTok{ }
\StringTok{  }\KeywordTok{gt}\NormalTok{()}
\end{Highlighting}
\end{Shaded}

\captionsetup[table]{labelformat=empty,skip=1pt}
\begin{longtable}{r}
\toprule
correlation \\ 
\midrule
0.857142 \\ 
\bottomrule
\end{longtable}

\hypertarget{f-for-all-other-predictors-held-constant-what-would-be-the-difference-in-predicted-expenditure-on-gambling-for-a-male-compared-to-a-female}{%
\subsubsection{(f) For all other predictors held constant, what would be
the difference in predicted expenditure on gambling for a male compared
to a
female?}\label{f-for-all-other-predictors-held-constant-what-would-be-the-difference-in-predicted-expenditure-on-gambling-for-a-male-compared-to-a-female}}

.

\begin{Shaded}
\begin{Highlighting}[]
\KeywordTok{data.frame}\NormalTok{(}\DataTypeTok{Gender_coef =}\NormalTok{ tg_lm}\OperatorTok{$}\NormalTok{coefficients[}\StringTok{"sex"}\NormalTok{]) }\OperatorTok\StringTok{ }
\StringTok{  }\KeywordTok{gt}\NormalTok{()}
\end{Highlighting}
\end{Shaded}

\captionsetup[table]{labelformat=empty,skip=1pt}
\begin{longtable}{r}
\toprule
Gender\_coef \\ 
\midrule
-22.11833 \\ 
\bottomrule
\end{longtable}

\hypertarget{chapter-3-interference}{%
\subsection{Chapter 3 Interference}\label{chapter-3-interference}}

\hypertarget{problem-1-1}{%
\subsection{Problem 1}\label{problem-1-1}}

\emph{For the prostate data, fit a model with lpsa as the response and
the other variables as predictors.}

\begin{Shaded}
\begin{Highlighting}[]
\KeywordTok{data}\NormalTok{(prostate)}
\KeywordTok{head}\NormalTok{(prostate)}
\end{Highlighting}
\end{Shaded}

\begin{verbatim}
##       lcavol lweight age      lbph svi      lcp gleason pgg45     lpsa
## 1 -0.5798185  2.7695  50 -1.386294   0 -1.38629       6     0 -0.43078
## 2 -0.9942523  3.3196  58 -1.386294   0 -1.38629       6     0 -0.16252
## 3 -0.5108256  2.6912  74 -1.386294   0 -1.38629       7    20 -0.16252
## 4 -1.2039728  3.2828  58 -1.386294   0 -1.38629       6     0 -0.16252
## 5  0.7514161  3.4324  62 -1.386294   0 -1.38629       6     0  0.37156
## 6 -1.0498221  3.2288  50 -1.386294   0 -1.38629       6     0  0.76547
\end{verbatim}

\begin{Shaded}
\begin{Highlighting}[]
\NormalTok{ps_lm <-}\StringTok{ }\KeywordTok{lm}\NormalTok{(lpsa }\OperatorTok{~}\StringTok{ }\NormalTok{lcavol }\OperatorTok{+}\StringTok{ }\NormalTok{lweight }\OperatorTok{+}\StringTok{ }\NormalTok{age }\OperatorTok{+}\StringTok{ }\NormalTok{lbph }\OperatorTok{+}\StringTok{ }\NormalTok{svi }\OperatorTok{+}\StringTok{ }\NormalTok{lcp }\OperatorTok{+}\StringTok{ }\NormalTok{gleason }\OperatorTok{+}\StringTok{ }\NormalTok{pgg45, }\DataTypeTok{data =}\NormalTok{ prostate)}
\NormalTok{ps_lms <-}\StringTok{ }\KeywordTok{summary}\NormalTok{(ps_lm)}
\NormalTok{ps_lms}
\end{Highlighting}
\end{Shaded}

\begin{verbatim}
## 
## Call:
## lm(formula = lpsa ~ lcavol + lweight + age + lbph + svi + lcp + 
##     gleason + pgg45, data = prostate)
## 
## Residuals:
##     Min      1Q  Median      3Q     Max 
## -1.7331 -0.3713 -0.0170  0.4141  1.6381 
## 
## Coefficients:
##              Estimate Std. Error t value Pr(>|t|)    
## (Intercept)  0.669337   1.296387   0.516  0.60693    
## lcavol       0.587022   0.087920   6.677 2.11e-09 ***
## lweight      0.454467   0.170012   2.673  0.00896 ** 
## age         -0.019637   0.011173  -1.758  0.08229 .  
## lbph         0.107054   0.058449   1.832  0.07040 .  
## svi          0.766157   0.244309   3.136  0.00233 ** 
## lcp         -0.105474   0.091013  -1.159  0.24964    
## gleason      0.045142   0.157465   0.287  0.77503    
## pgg45        0.004525   0.004421   1.024  0.30886    
## ---
## Signif. codes:  0 '***' 0.001 '**' 0.01 '*' 0.05 '.' 0.1 ' ' 1
## 
## Residual standard error: 0.7084 on 88 degrees of freedom
## Multiple R-squared:  0.6548, Adjusted R-squared:  0.6234 
## F-statistic: 20.86 on 8 and 88 DF,  p-value: < 2.2e-16
\end{verbatim}

\hypertarget{a-compute-90-and-95-cis-for-the-parameter-associated-with-age.-using-just-these-intervals-what-could-we-have-deduced-about-the-p-value-for-age-in-the-regression-summary}{%
\subsubsection{(a) Compute 90 and 95\% CIs for the parameter associated
with age. Using just these intervals, what could we have deduced about
the p-value for age in the regression
summary?}\label{a-compute-90-and-95-cis-for-the-parameter-associated-with-age.-using-just-these-intervals-what-could-we-have-deduced-about-the-p-value-for-age-in-the-regression-summary}}

/

\begin{Shaded}
\begin{Highlighting}[]
\KeywordTok{confint}\NormalTok{(ps_lm, }\DataTypeTok{parm =} \StringTok{"age"}\NormalTok{, }\DataTypeTok{level =} \FloatTok{0.90}\NormalTok{)}
\end{Highlighting}
\end{Shaded}

\begin{verbatim}
##            5 %         95 %
## age -0.0382102 -0.001064151
\end{verbatim}

\begin{Shaded}
\begin{Highlighting}[]
\KeywordTok{confint}\NormalTok{(ps_lm, }\DataTypeTok{parm =} \StringTok{"age"}\NormalTok{, }\DataTypeTok{level =} \FloatTok{0.95}\NormalTok{)}
\end{Highlighting}
\end{Shaded}

\begin{verbatim}
##           2.5 %      97.5 %
## age -0.04184062 0.002566267
\end{verbatim}

\begin{Shaded}
\begin{Highlighting}[]
\KeywordTok{names}\NormalTok{(ps_lms)}
\end{Highlighting}
\end{Shaded}

\begin{verbatim}
##  [1] "call"          "terms"         "residuals"     "coefficients" 
##  [5] "aliased"       "sigma"         "df"            "r.squared"    
##  [9] "adj.r.squared" "fstatistic"    "cov.unscaled"
\end{verbatim}

\begin{Shaded}
\begin{Highlighting}[]
\NormalTok{ps_lms}\OperatorTok{$}\NormalTok{coefficients[}\StringTok{"age"}\NormalTok{, }\StringTok{"Pr(>|t|)"}\NormalTok{]}
\end{Highlighting}
\end{Shaded}

\begin{verbatim}
## [1] 0.08229321
\end{verbatim}

\hypertarget{b-compute-and-display-a-95-joint-confidence-region-for-the-parameters-associated-with-age-and-lbph.-plot-the-origin-on-this-display.-the-location-of-the-origin-on-the-display-tells-us-the-outcome-of-a-certain-hypothesis-test.-state-that-test-and-its-outcome.}{%
\subsubsection{(b) Compute and display a 95\% joint confidence region
for the parameters associated with age and lbph. Plot the origin on this
display. The location of the origin on the display tells us the outcome
of a certain hypothesis test. State that test and its
outcome.}\label{b-compute-and-display-a-95-joint-confidence-region-for-the-parameters-associated-with-age-and-lbph.-plot-the-origin-on-this-display.-the-location-of-the-origin-on-the-display-tells-us-the-outcome-of-a-certain-hypothesis-test.-state-that-test-and-its-outcome.}}

/

\begin{Shaded}
\begin{Highlighting}[]
\KeywordTok{library}\NormalTok{(ellipse) }
\KeywordTok{plot}\NormalTok{(}\KeywordTok{ellipse}\NormalTok{(ps_lm, }\KeywordTok{c}\NormalTok{(}\StringTok{"age"}\NormalTok{, }\StringTok{"lbph"}\NormalTok{)), }\DataTypeTok{type =} \StringTok{"l"}\NormalTok{) }
\KeywordTok{points}\NormalTok{(}\DecValTok{0}\NormalTok{, }\DecValTok{0}\NormalTok{, }\DataTypeTok{pch =} \DecValTok{1}\NormalTok{) }
\KeywordTok{abline}\NormalTok{(}\DataTypeTok{v =} \KeywordTok{confint}\NormalTok{(ps_lm)[}\StringTok{'age'}\NormalTok{, ], }\DataTypeTok{lty =} \DecValTok{2}\NormalTok{) }
\KeywordTok{abline}\NormalTok{(}\DataTypeTok{h =} \KeywordTok{confint}\NormalTok{(ps_lm)[}\StringTok{'lbph'}\NormalTok{, ], }\DataTypeTok{lty =} \DecValTok{2}\NormalTok{) }
\end{Highlighting}
\end{Shaded}

\includegraphics{Faraway_files/figure-latex/chap3-problem-1-b-1.pdf}

\hypertarget{c-suppose-a-new-patient-with-the-following-values-arrives}{%
\subsubsection{(c) Suppose a new patient with the following values
arrives:}\label{c-suppose-a-new-patient-with-the-following-values-arrives}}

`data.framelcavol lweight age Ibph svi lcp 1.44692 3.62301 65.00000
0.30010 0.00000 -0.79851 gleason pgg45 7.00000 15.00000

\hypertarget{predict-the-lpsa-for-this-patient-along-with-an-appropriate-95-ci.}{%
\subsubsection{Predict the lpsa for this patient along with an
appropriate 95\%
CI.}\label{predict-the-lpsa-for-this-patient-along-with-an-appropriate-95-ci.}}

/

\begin{Shaded}
\begin{Highlighting}[]
\NormalTok{new_patient <-}\StringTok{ }\KeywordTok{data.frame}\NormalTok{(}
  \StringTok{"lcavol"}\NormalTok{ =}\StringTok{ }\FloatTok{1.44692}\NormalTok{,}
  \StringTok{"lweight"}\NormalTok{ =}\StringTok{ }\FloatTok{3.62301}\NormalTok{,}
  \StringTok{"age"}\NormalTok{ =}\StringTok{ }\FloatTok{65.00000}\NormalTok{,}
  \StringTok{"lbph"}\NormalTok{ =}\StringTok{ }\FloatTok{0.30010}\NormalTok{,}
  \StringTok{"svi"}\NormalTok{ =}\StringTok{ }\FloatTok{0.00000}\NormalTok{,}
  \StringTok{"lcp"}\NormalTok{ =}\StringTok{ }\FloatTok{-0.79851}\NormalTok{,}
  \StringTok{"gleason"}\NormalTok{ =}\StringTok{ }\FloatTok{7.00000}\NormalTok{,}
  \StringTok{"pgg45"}\NormalTok{ =}\StringTok{ }\FloatTok{15.00000}
\NormalTok{)}

\NormalTok{new_patient}
\end{Highlighting}
\end{Shaded}

\begin{verbatim}
##    lcavol lweight age   lbph svi      lcp gleason pgg45
## 1 1.44692 3.62301  65 0.3001   0 -0.79851       7    15
\end{verbatim}

\begin{Shaded}
\begin{Highlighting}[]
\KeywordTok{predict}\NormalTok{(ps_lm, }\DataTypeTok{newdata =}\NormalTok{ new_patient, }\DataTypeTok{interval =} \StringTok{"prediction"}\NormalTok{)}
\end{Highlighting}
\end{Shaded}

\begin{verbatim}
##        fit       lwr      upr
## 1 2.389053 0.9646584 3.813447
\end{verbatim}

\hypertarget{d-repeat-the-last-question-for-a-patient-with-the-same-values-except-that-he-or-she-is-age-20.-explain-why-the-ci-is-wider.}{%
\subsubsection{(d) Repeat the last question for a patient with the same
values except that he or she is age 20. Explain why the CI is
wider.}\label{d-repeat-the-last-question-for-a-patient-with-the-same-values-except-that-he-or-she-is-age-20.-explain-why-the-ci-is-wider.}}

/

\begin{Shaded}
\begin{Highlighting}[]
\NormalTok{new_patient2 <-}\StringTok{ }\NormalTok{new_patient}
\NormalTok{new_patient2[}\DecValTok{3}\NormalTok{] =}\StringTok{ }\DecValTok{20}
\KeywordTok{rbind}\NormalTok{(new_patient, new_patient2)}
\end{Highlighting}
\end{Shaded}

\begin{verbatim}
##    lcavol lweight age   lbph svi      lcp gleason pgg45
## 1 1.44692 3.62301  65 0.3001   0 -0.79851       7    15
## 2 1.44692 3.62301  20 0.3001   0 -0.79851       7    15
\end{verbatim}

\begin{Shaded}
\begin{Highlighting}[]
\KeywordTok{predict}\NormalTok{(ps_lm, }\DataTypeTok{newdata =}\NormalTok{ new_patient2, }\DataTypeTok{interval =} \StringTok{"prediction"}\NormalTok{)}
\end{Highlighting}
\end{Shaded}

\begin{verbatim}
##        fit      lwr      upr
## 1 3.272726 1.538744 5.006707
\end{verbatim}

\begin{Shaded}
\begin{Highlighting}[]
\KeywordTok{ggplot}\NormalTok{(}\DataTypeTok{data =}\NormalTok{ prostate, }\KeywordTok{aes}\NormalTok{(}\DataTypeTok{x =}\NormalTok{ age, }\DataTypeTok{y =}\NormalTok{ ..density..)) }\OperatorTok{+}
\StringTok{  }\KeywordTok{geom_histogram}\NormalTok{(}\DataTypeTok{bins =} \DecValTok{10}\NormalTok{, }\DataTypeTok{fill =} \StringTok{"steelblue"}\NormalTok{, }\DataTypeTok{colour =} \StringTok{"white"}\NormalTok{) }\OperatorTok{+}
\StringTok{  }\KeywordTok{ggtitle}\NormalTok{(}\DataTypeTok{label =} \StringTok{"Histogram of Age"}\NormalTok{) }\OperatorTok{+}
\StringTok{  }\KeywordTok{theme}\NormalTok{(}\DataTypeTok{plot.title =} \KeywordTok{element_text}\NormalTok{(}\DataTypeTok{size =} \DecValTok{15}\NormalTok{, }\DataTypeTok{hjust =} \FloatTok{0.5}\NormalTok{, }\DataTypeTok{vjust =} \FloatTok{1.5}\NormalTok{, }\DataTypeTok{face =} \StringTok{"bold"}\NormalTok{))}
\end{Highlighting}
\end{Shaded}

\includegraphics{Faraway_files/figure-latex/chap3-problem-1-d-1.pdf}

\hypertarget{e-in-the-text-we-made-a-permutation-test-corresponding-to-the-f-test-for-the-significance-of-all-the-predictors.-execute-the-permutation-test-corresponding-to-the-t-test-for-age-in-this-model.-hint-summary-g-coef-43-gets-you-the-t-statistic-you-need-if-the-model-is-called-g.}{%
\subsubsection{(e) In the text, we made a permutation test corresponding
to the F-test for the significance of all the predictors. Execute the
permutation test corresponding to the t-test for age in this model.
(Hint: \{summary (g) \$coef {[}4,3{]} gets you the t-statistic you need
if the model is called
g.)}\label{e-in-the-text-we-made-a-permutation-test-corresponding-to-the-f-test-for-the-significance-of-all-the-predictors.-execute-the-permutation-test-corresponding-to-the-t-test-for-age-in-this-model.-hint-summary-g-coef-43-gets-you-the-t-statistic-you-need-if-the-model-is-called-g.}}

/

\begin{Shaded}
\begin{Highlighting}[]
\NormalTok{t_value <-}\StringTok{ }\KeywordTok{summary}\NormalTok{(ps_lm) }\OperatorTok\StringTok{ }
\StringTok{  }\KeywordTok{coef}\NormalTok{() }\OperatorTok\StringTok{ }
\StringTok{  }\NormalTok{.[}\StringTok{"age"}\NormalTok{, }\StringTok{"t value"}\NormalTok{] }

\NormalTok{permute_tmod <-}\StringTok{ }\ControlFlowTok{function}\NormalTok{(nsims) \{}
  \KeywordTok{map_dbl}\NormalTok{(}\DecValTok{1}\OperatorTok{:}\NormalTok{nsims,}
          \OperatorTok{~}\StringTok{ }\KeywordTok{lm}\NormalTok{(}\KeywordTok{sample}\NormalTok{(lpsa) }\OperatorTok{~}\StringTok{ }\NormalTok{., }\DataTypeTok{data =}\NormalTok{ prostate) }\OperatorTok
\StringTok{            }\KeywordTok{summary}\NormalTok{() }\OperatorTok
\StringTok{            }\KeywordTok{coef}\NormalTok{() }\OperatorTok
\StringTok{            }\NormalTok{.[}\StringTok{"age"}\NormalTok{, }\StringTok{"t value"}\NormalTok{]) }
\NormalTok{\} }
  
\KeywordTok{mean}\NormalTok{(}\KeywordTok{abs}\NormalTok{(}\KeywordTok{permute_tmod}\NormalTok{(}\DecValTok{100}\NormalTok{)) }\OperatorTok{>}\StringTok{ }\KeywordTok{abs}\NormalTok{(t_value)) }
\end{Highlighting}
\end{Shaded}

\begin{verbatim}
## [1] 0.1
\end{verbatim}

\begin{Shaded}
\begin{Highlighting}[]
\KeywordTok{mean}\NormalTok{(}\KeywordTok{abs}\NormalTok{(}\KeywordTok{permute_tmod}\NormalTok{(}\DecValTok{1000}\NormalTok{)) }\OperatorTok{>}\StringTok{ }\KeywordTok{abs}\NormalTok{(t_value)) }
\end{Highlighting}
\end{Shaded}

\begin{verbatim}
## [1] 0.075
\end{verbatim}

\begin{Shaded}
\begin{Highlighting}[]
\KeywordTok{mean}\NormalTok{(}\KeywordTok{abs}\NormalTok{(}\KeywordTok{permute_tmod}\NormalTok{(}\DecValTok{10000}\NormalTok{)) }\OperatorTok{>}\StringTok{ }\KeywordTok{abs}\NormalTok{(t_value)) }
\end{Highlighting}
\end{Shaded}

\begin{verbatim}
## [1] 0.0825
\end{verbatim}

\end{document}
