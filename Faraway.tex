% Options for packages loaded elsewhere
\PassOptionsToPackage{unicode}{hyperref}
\PassOptionsToPackage{hyphens}{url}
%
\documentclass[
]{article}
\usepackage{lmodern}
\usepackage{amssymb,amsmath}
\usepackage{ifxetex,ifluatex}
\ifnum 0\ifxetex 1\fi\ifluatex 1\fi=0 % if pdftex
  \usepackage[T1]{fontenc}
  \usepackage[utf8]{inputenc}
  \usepackage{textcomp} % provide euro and other symbols
\else % if luatex or xetex
  \usepackage{unicode-math}
  \defaultfontfeatures{Scale=MatchLowercase}
  \defaultfontfeatures[\rmfamily]{Ligatures=TeX,Scale=1}
  \setmainfont[]{NanumGothic}
\fi
% Use upquote if available, for straight quotes in verbatim environments
\IfFileExists{upquote.sty}{\usepackage{upquote}}{}
\IfFileExists{microtype.sty}{% use microtype if available
  \usepackage[]{microtype}
  \UseMicrotypeSet[protrusion]{basicmath} % disable protrusion for tt fonts
}{}
\makeatletter
\@ifundefined{KOMAClassName}{% if non-KOMA class
  \IfFileExists{parskip.sty}{%
    \usepackage{parskip}
  }{% else
    \setlength{\parindent}{0pt}
    \setlength{\parskip}{6pt plus 2pt minus 1pt}}
}{% if KOMA class
  \KOMAoptions{parskip=half}}
\makeatother
\usepackage{xcolor}
\IfFileExists{xurl.sty}{\usepackage{xurl}}{} % add URL line breaks if available
\IfFileExists{bookmark.sty}{\usepackage{bookmark}}{\usepackage{hyperref}}
\hypersetup{
  pdftitle={Faraway},
  pdfauthor={Harry Woo},
  hidelinks,
  pdfcreator={LaTeX via pandoc}}
\urlstyle{same} % disable monospaced font for URLs
\usepackage[margin=1in]{geometry}
\usepackage{color}
\usepackage{fancyvrb}
\newcommand{\VerbBar}{|}
\newcommand{\VERB}{\Verb[commandchars=\\\{\}]}
\DefineVerbatimEnvironment{Highlighting}{Verbatim}{commandchars=\\\{\}}
% Add ',fontsize=\small' for more characters per line
\usepackage{framed}
\definecolor{shadecolor}{RGB}{248,248,248}
\newenvironment{Shaded}{\begin{snugshade}}{\end{snugshade}}
\newcommand{\AlertTok}[1]{\textcolor[rgb]{0.94,0.16,0.16}{#1}}
\newcommand{\AnnotationTok}[1]{\textcolor[rgb]{0.56,0.35,0.01}{\textbf{\textit{#1}}}}
\newcommand{\AttributeTok}[1]{\textcolor[rgb]{0.77,0.63,0.00}{#1}}
\newcommand{\BaseNTok}[1]{\textcolor[rgb]{0.00,0.00,0.81}{#1}}
\newcommand{\BuiltInTok}[1]{#1}
\newcommand{\CharTok}[1]{\textcolor[rgb]{0.31,0.60,0.02}{#1}}
\newcommand{\CommentTok}[1]{\textcolor[rgb]{0.56,0.35,0.01}{\textit{#1}}}
\newcommand{\CommentVarTok}[1]{\textcolor[rgb]{0.56,0.35,0.01}{\textbf{\textit{#1}}}}
\newcommand{\ConstantTok}[1]{\textcolor[rgb]{0.00,0.00,0.00}{#1}}
\newcommand{\ControlFlowTok}[1]{\textcolor[rgb]{0.13,0.29,0.53}{\textbf{#1}}}
\newcommand{\DataTypeTok}[1]{\textcolor[rgb]{0.13,0.29,0.53}{#1}}
\newcommand{\DecValTok}[1]{\textcolor[rgb]{0.00,0.00,0.81}{#1}}
\newcommand{\DocumentationTok}[1]{\textcolor[rgb]{0.56,0.35,0.01}{\textbf{\textit{#1}}}}
\newcommand{\ErrorTok}[1]{\textcolor[rgb]{0.64,0.00,0.00}{\textbf{#1}}}
\newcommand{\ExtensionTok}[1]{#1}
\newcommand{\FloatTok}[1]{\textcolor[rgb]{0.00,0.00,0.81}{#1}}
\newcommand{\FunctionTok}[1]{\textcolor[rgb]{0.00,0.00,0.00}{#1}}
\newcommand{\ImportTok}[1]{#1}
\newcommand{\InformationTok}[1]{\textcolor[rgb]{0.56,0.35,0.01}{\textbf{\textit{#1}}}}
\newcommand{\KeywordTok}[1]{\textcolor[rgb]{0.13,0.29,0.53}{\textbf{#1}}}
\newcommand{\NormalTok}[1]{#1}
\newcommand{\OperatorTok}[1]{\textcolor[rgb]{0.81,0.36,0.00}{\textbf{#1}}}
\newcommand{\OtherTok}[1]{\textcolor[rgb]{0.56,0.35,0.01}{#1}}
\newcommand{\PreprocessorTok}[1]{\textcolor[rgb]{0.56,0.35,0.01}{\textit{#1}}}
\newcommand{\RegionMarkerTok}[1]{#1}
\newcommand{\SpecialCharTok}[1]{\textcolor[rgb]{0.00,0.00,0.00}{#1}}
\newcommand{\SpecialStringTok}[1]{\textcolor[rgb]{0.31,0.60,0.02}{#1}}
\newcommand{\StringTok}[1]{\textcolor[rgb]{0.31,0.60,0.02}{#1}}
\newcommand{\VariableTok}[1]{\textcolor[rgb]{0.00,0.00,0.00}{#1}}
\newcommand{\VerbatimStringTok}[1]{\textcolor[rgb]{0.31,0.60,0.02}{#1}}
\newcommand{\WarningTok}[1]{\textcolor[rgb]{0.56,0.35,0.01}{\textbf{\textit{#1}}}}
\usepackage{graphicx,grffile}
\makeatletter
\def\maxwidth{\ifdim\Gin@nat@width>\linewidth\linewidth\else\Gin@nat@width\fi}
\def\maxheight{\ifdim\Gin@nat@height>\textheight\textheight\else\Gin@nat@height\fi}
\makeatother
% Scale images if necessary, so that they will not overflow the page
% margins by default, and it is still possible to overwrite the defaults
% using explicit options in \includegraphics[width, height, ...]{}
\setkeys{Gin}{width=\maxwidth,height=\maxheight,keepaspectratio}
% Set default figure placement to htbp
\makeatletter
\def\fps@figure{htbp}
\makeatother
\setlength{\emergencystretch}{3em} % prevent overfull lines
\providecommand{\tightlist}{%
  \setlength{\itemsep}{0pt}\setlength{\parskip}{0pt}}
\setcounter{secnumdepth}{-\maxdimen} % remove section numbering
\usepackage{amsmath}
\usepackage{booktabs}
\usepackage{caption}
\usepackage{longtable}

\title{Faraway}
\author{Harry Woo}
\date{2020-5-12}

\begin{document}
\maketitle

This is an R Markdown document. Markdown is a simple formatting syntax
for authoring HTML, PDF, and MS Word documents. For more details on
using R Markdown see \url{http://rmarkdown.rstudio.com}.

\hypertarget{problem-2.1}{%
\subsection{Problem 2.1}\label{problem-2.1}}

\emph{The dataset teengamb concerns a study of teenage gambling in
Britain. Fit a regression model with the expenditure on gambling as the
response and the sex, status, income and verbal score as predictors.
Present the output.}

\begin{Shaded}
\begin{Highlighting}[]
\NormalTok{tg_lm <-}\StringTok{  }\KeywordTok{lm}\NormalTok{(gamble }\OperatorTok{~}\StringTok{ }\NormalTok{sex }\OperatorTok{+}\StringTok{ }\NormalTok{status }\OperatorTok{+}\StringTok{ }\NormalTok{income }\OperatorTok{+}\StringTok{ }\NormalTok{verbal, }\DataTypeTok{data =}\NormalTok{ teengamb)}
\NormalTok{tg_lms <-}\StringTok{ }\KeywordTok{summary}\NormalTok{(tg_lm)}
\KeywordTok{print}\NormalTok{(tg_lms)}
\end{Highlighting}
\end{Shaded}

\begin{verbatim}
## 
## Call:
## lm(formula = gamble ~ sex + status + income + verbal, data = teengamb)
## 
## Residuals:
##     Min      1Q  Median      3Q     Max 
## -51.082 -11.320  -1.451   9.452  94.252 
## 
## Coefficients:
##              Estimate Std. Error t value Pr(>|t|)    
## (Intercept)  22.55565   17.19680   1.312   0.1968    
## sex         -22.11833    8.21111  -2.694   0.0101 *  
## status        0.05223    0.28111   0.186   0.8535    
## income        4.96198    1.02539   4.839 1.79e-05 ***
## verbal       -2.95949    2.17215  -1.362   0.1803    
## ---
## Signif. codes:  0 '***' 0.001 '**' 0.01 '*' 0.05 '.' 0.1 ' ' 1
## 
## Residual standard error: 22.69 on 42 degrees of freedom
## Multiple R-squared:  0.5267, Adjusted R-squared:  0.4816 
## F-statistic: 11.69 on 4 and 42 DF,  p-value: 1.815e-06
\end{verbatim}

\hypertarget{a-what-percentage-of-variation-in-the-response-is-explained-by-these-predictors}{%
\subsubsection{(a) What percentage of variation in the response is
explained by these
predictors?}\label{a-what-percentage-of-variation-in-the-response-is-explained-by-these-predictors}}

전체 제곱합(SST)에서 회귀 제곱합(SSR)이 설명하는 비중, 즉 모형의
설명력은 결정 계수 R\textsuperscript{2} 이다. 위 Summary 에서와 같이 동
모형의 결정계수 \textbf{Multiple R-squared = 0.5267} 이다.

\begin{Shaded}
\begin{Highlighting}[]
\NormalTok{var_ex <-}\StringTok{ }\KeywordTok{data.frame}\NormalTok{(}\DataTypeTok{Var_explained =}\NormalTok{ tg_lms}\OperatorTok{$}\NormalTok{r.squared)}
\NormalTok{var_ex }\OperatorTok\StringTok{ }\KeywordTok{gt}\NormalTok{() }\OperatorTok\StringTok{ }
\StringTok{  }\KeywordTok{fmt_percent}\NormalTok{(}\DataTypeTok{columns =} \KeywordTok{vars}\NormalTok{(Var_explained),}
              \DataTypeTok{decimals =} \DecValTok{2}\NormalTok{)}
\end{Highlighting}
\end{Shaded}

\captionsetup[table]{labelformat=empty,skip=1pt}
\begin{longtable}{r}
\toprule
Var\_explained \\ 
\midrule
$52.67\%$ \\ 
\bottomrule
\end{longtable}

\hypertarget{b-which-observation-has-the-largest-positive-residual-give-the-case-number.}{%
\subsubsection{(b) Which observation has the largest (positive)
residual? Give the case
number.}\label{b-which-observation-has-the-largest-positive-residual-give-the-case-number.}}

회귀모형의 \texttt{residuals} 를 데이터프레임으로 변환하여 잔차값
기준으로 내림차순 정렬을 시행해 largest residual의 case number를 추출한
결과, \textbf{해당 case number는 24} 이다.

\begin{Shaded}
\begin{Highlighting}[]
\NormalTok{res <-}\StringTok{ }\KeywordTok{data.frame}\NormalTok{(}\DataTypeTok{case_no =} \KeywordTok{c}\NormalTok{(}\DecValTok{1}\OperatorTok{:}\DecValTok{47}\NormalTok{), }\DataTypeTok{residual =}\NormalTok{ tg_lm}\OperatorTok{$}\NormalTok{residuals)}
\NormalTok{res }\OperatorTok\StringTok{ }
\StringTok{  }\KeywordTok{arrange}\NormalTok{(}\KeywordTok{desc}\NormalTok{(residual)) }\OperatorTok\StringTok{ }
\StringTok{  }\KeywordTok{slice}\NormalTok{(}\DecValTok{1}\NormalTok{) }\OperatorTok\StringTok{ }
\StringTok{  }\KeywordTok{gt}\NormalTok{()}
\end{Highlighting}
\end{Shaded}

\captionsetup[table]{labelformat=empty,skip=1pt}
\begin{longtable}{cr}
\toprule
case\_no & residual \\ 
\midrule
24 & 94.25222 \\ 
\bottomrule
\end{longtable}

\hypertarget{including-plots}{%
\subsection{Including Plots}\label{including-plots}}

You can also embed plots, for example:

\includegraphics{Faraway_files/figure-latex/pressure-1.pdf}

Note that the \texttt{echo\ =\ FALSE} parameter was added to the code
chunk to prevent printing of the R code that generated the plot.

\end{document}
